% !TEX program = xelatex
\documentclass{article}
\usepackage{ctex}
\newtheorem{definition}{定义}[section]
\newtheorem{theorem}{定理}[section]
\newtheorem{lemma}[theorem]{引理}
\newtheorem{corollary}[theorem]{推论}
\usepackage{indentfirst}
\setlength{\parindent}{2em}

\begin{document}
  \title{论浙江精神的选取以及作用和发扬}
  \author{姓名:***\quad 学号:***\\班级:***(***)***}
  \date{\today}
  \maketitle
  \begin{abstract}
    改革开放以来,浙江省经济一路上升,位居全国前列。作为浙江省人民的内驱力,浙江精神发挥着至关重要的作用。对人民的品格,对经济的发展,都是必不可少的。而浙江精神的提炼,也十分细致,符合着浙江这个地区从古至今的绝大多数人们的精神。
  \end{abstract}
  \section{什么是浙江精神}
  2000年,浙江提炼出的浙江精神是16个字:“自强不息、坚韧不拔、勇于创新、讲求实效”。而到了2006年,浙江又把浙江精神再次提炼为12个字:“求真务实、诚信和谐、开放图强”。“民族精神是一个民族赖以生存和发展的精神支撑。”精神一词,在历史上多次展现了它的魅力。屈原投湖自尽,是爱国精神的支撑。勾践卧薪尝胆,是自强精神的支撑。时间长河在流逝,精神也在更新换代,红船精神,长征精神,奥运精神,塞罕坝精神,女排精神。每一个精神都值得赞颂。每一个精神,都是一群人的抽象化。到了现在,浙江精神,那就是对浙江人的赞颂,也是对浙江人的要求和鞭策。那浙江精神是什么?为什么选这些作为浙江精神呢?

  \subsection{求真务实}
  求真务实曾在中纪委第三次全体会议上提出过。胡锦涛同志曾指出的:我们党80多年的历程充分说明,求真务实是党的活力之所在,也是党和人民事业兴旺发达的关键之所在。所以对浙江人来说,就是要踏踏实实做事,真真切切做人。而这是干好没见小事的基础,小事稳定,大事才能成。

  \subsection{诚信和谐}
  魏徵言:夫妇有恩矣,不诚则离。王安石言:自古驱民在信诚,一言为重百金轻。此乃诚信、和谐,则是与人为善,这是儒家的观点,是流淌在中华民族血脉里的东西。诚信和谐,不光是为了营造良好的社会风尚,更是为了创造优良的商业环境。 “商圣之地,事功之学,浙商之脉,源远流长。\footnote{《浙江精神与浙江发展》 第74页,《首届世界浙商大会宣言》。} 而这,正是诚信和谐才能换来。想要源远流长,必然要人与人之间的信任。

  \subsection{开放图强}
  重点在开放。中国能有现在的经济实力,与开放密不可分。在1759年到1842年,清朝采取了闭关锁国。而英法等国却在进行着工业革命。一边是科技飞速发展的西方世界,一边是对世界情况不做关注的清朝。短短几年,清朝的霸主地位就丢失了,科技不先进,跟本无法与英法等国进行对抗。随后的鸦片战争,八国联军侵华战争就发生了。而这仅仅因为闭关锁国,不接受外界的事物。而1978年,中国实行改革开放,40几年的时间,中国就成了世界第一大经济体。而浙江作为中国的高经济省份,开放当然必不可少。

  \section{浙江精神又有何作用?}
  \subsection{浙江精神代给了我们发展机遇}
  2016年,习近平总书记在“G20杭州峰会”结束之际,对浙江提出了秉持浙江精神的新要求:“干在实处、走在前列、勇立潮头”。在浙江精神之下,我们有干在实处的基础,有走在前列的品格,有勇立潮头的自信。从前县级市的义乌,变成了今日世界的义乌。一件件不起眼的小商品,在义乌人的手上,因求真务实有了质量,因诚信和谐有了顾客,因开放图强有了优势。浙江精神在他们身上展现的淋漓尽致。不妨说他们就是浙江精神。乌镇,被中国人叫作水乡。在别人看来,应该以旅游文明,但它以旅游文明了中国,以科技文明了世界。世界互联网大会的召开,证明了它的成功,更证明了浙江精神作用。  在浙江精神的领导指引下,浙商经济充满活力和无限生机。\footnote{《浙江精神与浙江发展》 第77页。}

  \subsection{浙江精神培育了良好品格}
  浙江东阳的邵飘萍,是中国近代新闻史上的著名报人,《京报》创办者,革命烈士。今浙江绍兴的秋瑾,中国民主主义革命活动家,妇女解放运动先驱。刘伯温,元末明初政治家,今浙江省丽水青田人,诗文古朴雄放,不乏抨击统治者腐朽,同 情民间疾苦之作。还有许许多多的名人,浙江精神在他们身上展现,自古就接触他们作品的浙江人,从小就受这些品质的熏陶,必将养成优秀的品格。  “从东汉的王充到南宋的浙东学派,从明代的阳明心学到明清之际王宗羲的经史之学,乃至近代的龚自珍、章太炎、鲁迅等,他们都对现实社会有着强烈的关怀意识。\footnote{《浙江精神和浙江发展》 第64页。}所以,每一个身处浙江的人,在浙江长大的人,都受着名人的熏陶,浙江精神的感染。都有着优良的品格。

  \subsection{浙江人民该如何继续发扬浙江精神呢?}
  古有孔子三千弟子,当时并没有现在这样先进的通信手段。孔子的弟子,都是靠他自己高尚的言行举止获得大家的认可的。所以,浙江精神并没有什么不好的地方,仅仅只是人们完整的展现他就可以获得别人的认可。但其中的困难之处是,人无完人。人总会遇到自己无法控制的情况,从而引发别人的不解或不满。但这并非是浙江精神的问题 。因此,既然无法避免,那就坦然面对。浙江人要有文化自信,而浙江精神有着深厚的文化底蕴。  “在漫长的历史实践过程中,从大禹的因势利导、敬业治水,到勾践的卧薪尝胆、励精图治;从钱氏的保境安民、纳土归宋,到胡则的为官一任、造福一方;从岳飞、于谦的精忠报国、清白一生,到方孝孺、张苍水的刚正不阿、以身殉国;从沈括的博学多识、精研深究,到竺可桢的科学救国、求是一生;无论是陈亮、叶适的经世致用,还是黄宗羲的工商皆本;无论是王充、王阳明的批判、自觉、还是龚自珍、蔡元培的开明、开放;无论是百年老店胡庆余堂的戒欺、诚信,还是宁波 、湖州商人的勤勉、善举;等等,都是浙江精神奠定了深厚的文化底蕴。浙江精神得以凝练成了以人为本、注重民生的观念,求真务实、主题自觉的理性,兼容并蓄、创业创新的胸襟,人我共生。天人合一的情怀,讲义守信。义利并举的品行,刚健正直、坚贞不屈的气节和卧薪尝胆、发奋图强的志向。\footnote{《浙江精神与浙江发展》 第55页。}此乃文化自信和精神自信的基石。因此,有着如此厚重的文化,凝实的精神,浙江精神将不会没落。
\end{document}
